\section*{Why I can not use Prezi\texttrademark{}}
\subsection*{Introduction}
\paragraph{}
Prezi\texttrademark{} the `freemuim' Zooming User Interface (ZUI) presentation
composition software. Requires the `user' to make an account with
Prezi\texttrademark{} in order for the `user' to utilize the service. To make an
account with Prezi\texttrademark{} the `user' has to agree to the Terms of
Service (TOS) and privacy policy. In the case the `user' does not agree with the
TOS, the `user' can not make an account with Prezi\texttrademark{} as such, the
`user' can not utilize the Prezi\texttrademark{} software.
\textit{\textbf{\underline{NOTE}}} the `user' is 100\% compliant with doin the
assignment on the `user's' own terms.
\par

\subsection{Licenses}
\paragraph{}
In \S6.2 \P 2 of the Prezi\texttrademark{} TOS it is stated ``With
respect to Public User Content, you hereby do and shall grant to Prezi (and its
successors, assigns, and third party service providers) a worldwide,
non-exclusive, revocable, royalty-free, fully paid, sublicensable, and
transferable license to use, host, store, reproduce, modify, create derivative
works, communicate, publish, publicly perform, publicly display, distribute and
transmit the content (1) for the purpose of providing you, and those with whom
you have shared your presentations (including the public), with the Service;
and (2) in connection with promotion and marketing of Prezi products and
services, including without limitation allowing third parties to search or index
the content, in connection with email promotions, product demonstrations, and
the like. This license ends when you delete your Public User Content or your
account is closed (either by you or by us), except (i) to the extent that your
Public User Content has been shared with others and they have not deleted it
and (ii) that we retain a license to maintain a back-up copy of your Public User
Content indefinitely.'' the shady and dodgy tactics of Prezi\texttrademark{}
keeping content indefinitely even after deletion, and commercial use of `user'
content in order to promote or demonstrate their own software. Also the
practically full and utter control they take of the content, as stated above in
lines 1--6. \textit{\textbf{\underline{NO}}} permission is asked of the
`user' (with the exception of agreeing or not agreeing to the TOS) whether they
want Prezi\texttrademark{} to keep the content after deletion, use it in a
commercial aspect, and give said granted permissions, licenses, etc to
Prezi\texttrademark{}, it's succcessors, third-parties, etc, etc. Instead the
company and it's affiliates rely on strong-arm appauling mafia-like tactics to
make the user submit to their terms and their terms only. The sense of personal
ownership is utterly destroyed in \S6.2 \P 2 of the Prezi\texttrademark{} TOS as
(I've stated exhaustively) Prezi\texttrademark{} and affiliates can use the
`user's' content however they see fit, only giving the `user' minimal control
of their content in reality.
\par

%\newpage
\subsection{General Agreement}
\paragraph{}
In \S16 \P 1--4 ``General Agreement'' of the Prezi\texttrademark{} TOS it states
``Governing Law. The laws of the state of California, excluding California’s
conflict of laws rules, will apply to any disputes arising out of or relating to
these terms or the Prezi Service.
\\ \\
Dispute Resolution. Any dispute arising out of or relating to these terms or the
Prezi Service shall be submitted exclusively to confidential binding arbitration
in San Francisco, California, except that to the extent you have in any manner
violated or threatened to violate Prezi’s intellectual property rights, we may
seek injunctive or other appropriate relief in any state or federal court in the
State of California. You hereby consent to, and waive all defenses of lack of
personal jurisdiction and forum non conveniens with respect to venue and
jurisdiction in the state and federal courts of California. Arbitration under
these Terms of Use shall be conducted pursuant to the Commercial Arbitration
Rules then prevailing at the American Arbitration Association. The arbitrator’s
award shall be final and binding and may be entered as a judgment in any court
of competent jurisdiction. To the fullest extent permitted by applicable law,
no arbitration under this Agreement shall be joined to an arbitration involving
any other party subject to this Agreement, whether through class action
proceedings or otherwise.
\\ \\
Statute of Limitations. You agree that regardless of any statue or law to the
contrary, any claim under this Agreement must be brought within one (1) year
after the cause of action arises, or such claim or cause of action is forever
barred.'' Let us break this down paragraph by paragraph.
\par
\newpage

\subsubsection{\S16 \P1}
\paragraph{}
Paragraph one is a general disclosure and disclaimer in a way of laws that do
and don't apply.
\par

\subsubsection{\S16 \P2}
\paragraph{}
Paragraph 2 of section 16 clearly outlines the avoidance of a
litigation case by Prezi\texttrademark{} in favor of arbitration which in the past
big companies used arbitration to control the outcome, in fact many present-day
services now sway towards arbitration simply because of a more controlling
environment for them. In fact in \S 16 \P 2 arbitration is on a purely
Prezi\texttrademark{} vs.\ individual basis as no class action proceedings are
permitted.
\par

\subsubsection{\S16 \P3}
\paragraph{} \S16 \P3 is a simple disclosure/disclaimer again about statue of
limitations, with the added exception of the complete disregarding of the statue
of limitations, making the `user' file a complaint in a period of 1 year after
``the cause of action arises''.
\par

%\newpage
\subsection{Electronic Communications Privacy Act Notice (18 U.S.C. 2701--2711)
of the desktop-EULA}
\paragraph{}
in the Prezi\texttrademark{} desktop End User License Agreement (EULA) it is
stated in \S8 \P1 ``Prezi makes no guarantee regarding the confidentiality or
privacy of any communication or information transmitted using the software or
the PPrezi service. Prezi will not be liable for the privacy of email addresses,
registration and identification information, disk space, communications,
confidential or trade-secret information, or any other Content stored on
equipment, transmitted over networks accessed by the Software or the Prezi
Service, or otherwise connected with your use of the Software or the Prezi
Service.'' This hand in hand with \S6.2 \P2 of the Prezi\texttrademark{} TOS
leaves room for massive abuse of the `user's' data. I.E. Prezi\texttrademark{}
would not be liable if they were to hand out content that has been uploaded to
their service, E.G. (Presentations, pictures, music clips, video clips, etc,
etc). Effectively making Prezi\texttrademark{} able to hand out a `user's' data at
their own discretion and hold it indefinitely.
\par

\subsection{Source and Security}
\paragraph{}
Prezi\texttrademark{} is developed in Adobe Flash, Adobe AIR, and built on the
Django web framework, also for those without Adobe Flash Prezi\texttrademark{}
offers a javascript based version. Adobe Flash is flawed with gaping security
holes that could pose a risk to the school's computers or the `user's' computer.
I won't even enurmerate on the various vulnerabilities and security flaws in
Adobe Flash so instead I will just leave this link,
\href{http://www.cvedetails.com/vulnerability-list/vendor_id-53/product_id-6761/Adobe-Flash-Player.html}{CVE details for Adobe Flash}.
This is a more non-techy friendly article that goes over the CVE details data,
\href{http://www.veracode.com/security/flash-security}{Veracode Flash-security}
\par

\subsection{The `user' is just plain being forced!}
\paragraph{}
The reason of why this document was written is because the `user' is being
forced to agree to a contract (the TOS and EULA) to make an account to utilize
the service the `user' does not want to utilize.
\par
